\section{Week 1}
\subsection{Q: Show that $d$ is coordinate invariant}
During the first lecture we defined $d$ as
\begin{equation}
\rd \equiv \frac{\partial}{\partial x_{\mu}} \id x^{\mu}.
\end{equation}
By changing basis of the contravariant $dx^{\mu}$ tensor using the chain rule we get
\begin{equation}
\rd x^{\mu} = \frac{\partial x^{\mu}}{\partial y_{\nu}} \id y^{\nu}.
\end{equation}
We can now rewrite our expression for $\rd$ as
\begin{equation}
\rd = \frac{\partial}{\partial x_{\mu}} \id x^{\mu} =\frac{\partial}{\partial x_{\mu}} \frac{\partial x^{\mu}}{\partial y_{\nu}} \id y^{\nu}.
\end{equation}
However, by using the chaing rule we recognize that this is nothing else than
\begin{equation}
\frac{\partial}{\partial y_{\nu}} \id y^{\nu}.
\end{equation}
Since $y$ and $x$ corresponds to different coordinatesystems the invariance of the operator $\rd$ is shown.
\cite{tidigarekandidat}