\subsection{Q: Show $\mathfrak{so}(3)\cong \mathfrak{su}(2)$}

The generators $T^i$ for SO(3) can be represented by the matrices

\begin{equation*}
T^1 \equiv T^{23} = 
\begin{bmatrix} 0 & 0 & 0 \\
 0 & 0 & 1 \\
  0 & -1 & 0 
  \end{bmatrix},\;\;\;
T^2  \equiv T^{31}=
\begin{bmatrix} 0 & 0 & 1 \\
 0 & 0 & 0 \\
  -1 & 0 & 0 
  \end{bmatrix},\;\;\;
T^3 \equiv T^{12}=
\begin{bmatrix} 0 & 1 & 0 \\
 -1 & 0 & 0 \\
  0 & 0 & 0 
\end{bmatrix}
\end{equation*}
or equivalently

$$
(T^{\alpha \beta})_{cd}= {\delta}^\alpha_c {\delta}^\beta_d-{\delta}^\alpha_d {\delta}^\beta_c=\epsilon_{cd} \epsilon^{\alpha \beta}
.$$

Now a simple calculation gives us

\begin{gather*}
\begin{bmatrix}
T^1 , T^2
\end{bmatrix} =
\begin{bmatrix} 0 & 0 & 0 \\
 0 & 0 & 1 \\
  0 & -1 & 0 
  \end{bmatrix}
  \cdot
  \begin{bmatrix} 0 & 0 & 1 \\
 0 & 0 & 0 \\
  -1 & 0 & 0 
  \end{bmatrix}
  -
  \begin{bmatrix} 0 & 0 & 1 \\
 0 & 0 & 0 \\
  -1 & 0 & 0 
  \end{bmatrix}
  \cdot
  \begin{bmatrix} 0 & 0 & 0 \\
 0 & 0 & 1 \\
  0 & -1 & 0 
  \end{bmatrix} =\\
  \begin{bmatrix} 0 & 0 & 0 \\
 -1 & 0 & 0 \\
  0 & 0 & 0 
  \end{bmatrix} -
  \begin{bmatrix} 0 & -1 & 0 \\
 0 & 0 & 0 \\
  0 & 0 & 0 
  \end{bmatrix} = 
  \begin{bmatrix} 0 & 1 & 0 \\
 -1 & 0 & 0 \\
  0 & 0 & 0 
  \end{bmatrix} =T^3
\end{gather*}

\begin{gather*}
\begin{bmatrix}
T^2 , T^3
\end{bmatrix} =
\begin{bmatrix} 0 & 0 & 1 \\
 0 & 0 & 0 \\
  -1 & 0 & 0 
  \end{bmatrix}
  \cdot
  \begin{bmatrix} 0 & 1 & 0 \\
 -1 & 0 & 0 \\
  0 & 0 & 0 
  \end{bmatrix}
  -
  \begin{bmatrix} 0 & 1 & 0 \\
 -1 & 0 & 0 \\
  0 & 0 & 0 
  \end{bmatrix}
  \cdot
  \begin{bmatrix} 0 & 0 & 1 \\
 0 & 0 & 0 \\
  -1 & 0 & 0 
  \end{bmatrix} =\\
  \begin{bmatrix} 0 & 0 & 0 \\
 0 & 0 & 0 \\
  0 & -1 & 0 
  \end{bmatrix} -
  \begin{bmatrix} 0 & 0 & 0 \\
 0 & 0 & -1 \\
  0 & 0 & 0 
  \end{bmatrix} = 
  \begin{bmatrix} 0 & 0 & 0 \\
 0 & 0 & 1 \\
  0 & -1 & 0 
  \end{bmatrix} =T^1.
\end{gather*}

\begin{gather*}
\begin{bmatrix}
T^3 , T^1
\end{bmatrix} =
\begin{bmatrix} 0 & 1 & 0 \\
 -1 & 0 & 0 \\
  0 & 0 & 0 
  \end{bmatrix}
  \cdot
  \begin{bmatrix} 0 & 0 & 0 \\
 0 & 0 & 1 \\
  0 & -1 & 0 
  \end{bmatrix}
  -
  \begin{bmatrix} 0 & 0 & 0 \\
 0 & 0 & 1 \\
  0 & -1 & 0 
  \end{bmatrix}
  \cdot
  \begin{bmatrix} 0 & 1 & 0 \\
 -1 & 0 & 0 \\
  0 & 0 & 0 
  \end{bmatrix} =\\
  \begin{bmatrix} 0 & 0 & 1 \\
 0 & 0 & 0 \\
  0 & 0 & 0 
  \end{bmatrix} -
  \begin{bmatrix} 0 & 0 & 0 \\
 0 & 0 & 0 \\
  1 & 0 & 0 
  \end{bmatrix} = 
  \begin{bmatrix} 0 & 0 & 1 \\
 0 & 0 & 0 \\
  -1 & 0 & 0 
  \end{bmatrix} =T^1
\end{gather*}

which can be written as

\begin{equation*}
[T^i,T^j]=\epsilon_{ijk}T^k.
\end{equation*}

But this is precisely the Lie algebra for SU(2) which was to be shown.