
\subsection{Q: Solve $\Lambda\Lambda\eta=\eta$ for $\varepsilon(x)$ where $\Lambda=\delta+\frac{\partial \varepsilon}{\partial x}+\mathcal{O}(\varepsilon^2)$}\label{sec:A24}

$$\Lambda\ein{\mu}{\nu}=\pder{\tilde{x}^\mu}{x^\nu}$$

$$\tilde{x}^\mu=x^\mu+\varepsilon^\mu(x)$$

$$\Lambda\ein{\mu}{\nu}=\delta^\mu_\nu+\pder{\varepsilon^\mu}{x^\nu}$$

$$\pder{\Lambda\ein{\mu}{\nu}}{x^{\gamma}}=\ppder{\tilde{x}^\mu}{x^\nu}{x^\gamma}=\ppder{\varepsilon^\mu}{x^\nu}{x^\gamma}$$



$$\Lambda\ein{\mu}{\nu}\Lambda\ein{\rho}{\gamma}\eta_{\mu\rho}=\eta_{\nu\gamma}$$


$$\pder{\tilde{x}^\mu}{x^\nu}\,\pder{\tilde{x}^\rho}{x^\gamma}\eta_{\mu\rho}=\eta_{\nu\gamma}$$


$$
\ppder{\tilde{x}^\mu}{x^\nu}{x^\kappa}\,\pder{\tilde{x}^\rho}{x^\gamma}\eta_{\mu\rho}
+\pder{\tilde{x}^\mu}{x^\nu}\,\ppder{\tilde{x}^\rho}{x^\gamma}{x^\kappa}\eta_{\mu\rho}=0
$$

but as

$$\eta_{\mu\rho}=\eta_{\rho\mu}$$

and that we contract with $\mu$ and $\rho$ that $\eta_{\mu\rho}$ has the requirment

\[\eta_{\mu\nu} = \left\{
  \begin{array}{lr}
    \pm1 & $if $ \nu=\mu\\
    0 & $if $ \nu\neq\mu
  \end{array}
\right.
\]

gives

$$\ppder{\tilde{x}^\mu}{x^\nu}{x^\kappa}\,\pder{\tilde{x}^\rho}{x^\gamma}\eta_{\mu\rho}=\pder{\tilde{x}^\mu}{x^\nu}\,\ppder{\tilde{x}^\rho}{x^\gamma}{x^\kappa}\eta_{\mu\rho}$$

thus


$$
2\ppder{\tilde{x}^\mu}{x^\nu}{x^\kappa}\,\pder{\tilde{x}^\rho}{x^\gamma}\eta_{\mu\rho}=0
$$

but as

$$\pder{\tilde{x}^\rho}{x^\gamma}=\Lambda\ein{\rho}{\gamma}\neq0$$

We conclude that 

$$\ppder{\tilde{x}^\mu}{x^\nu}{x^\kappa}=0$$

This means that

$$\ppder{\varepsilon^\mu}{x^\nu}{x^\gamma}=\ppder{\tilde{x}^\mu}{x^\nu}{x^\gamma}=0$$

Thus $\varepsilon^\mu(x)$ has to be a linear equation

We can therefore write it as:

$$\varepsilon^{\mu}(x)=a^\mu+\omega\ein{\mu}{\nu}x^\nu$$

\subsubsection{Constructing the Lie algebra for the Poincare group}

before continuing we want to show that 

$$\omega_{\mu\nu}=-\omega_{\nu\mu}$$

but having defined $\varepsilon$ this way we find that 

$$\pder{\varepsilon^\mu}{x^\nu}=\partial_\nu(a^\mu+\omega\ein{\mu}{\gamma}x^{\gamma})=\omega\ein{\mu}{\gamma}\delta_{\gamma\nu}=\omega\ein{\mu}{\nu}$$

thus we can rewrite $\Lambda$ as

$$\Lambda\ein{\mu}{\nu}=\delta^\mu_\nu+\omega\ein{\mu}{\nu}$$

$$\Lambda\ein{\mu}{\nu}\Lambda\ein{\rho}{\gamma}\eta_{\mu\rho}=\eta_{\nu\gamma}$$

$$(\delta^\mu_\nu+\omega\ein{\mu}{\nu})(\delta^\rho_\gamma+\omega\ein{\rho}{\gamma})\eta_{\mu\rho}=\eta_{\nu\gamma}$$

$$(\delta^\mu_\nu\delta^\rho_\gamma+\delta^\mu_\nu\omega\ein{\rho}{\gamma}+\omega\ein{\mu}{\nu}\delta^\rho_\gamma+\omega\ein{\mu}{\nu}\omega\ein{\rho}{\gamma})\eta_{\mu\rho}=\eta_{\nu\gamma}$$


$$(\delta^\mu_\nu\omega\ein{\rho}{\gamma}+\omega\ein{\mu}{\nu}\delta^\rho_\gamma+\omega\ein{\mu}{\nu}\omega\ein{\rho}{\gamma})\eta_{\mu\rho}=0$$


$$\delta^\mu_\nu\omega\ein{\rho}{\gamma}\eta_{\mu\rho}
+\omega\ein{\mu}{\nu}\delta^\rho_\gamma\eta_{\mu\rho}
+\omega\ein{\mu}{\nu}\omega\ein{\rho}{\gamma}\eta_{\mu\rho}=0$$

$$\delta^\mu_\nu\omega_{\mu\gamma}+\omega_{\rho\nu}\delta^\rho_\gamma+\omega\ein{\mu}{\nu}\omega\ein{\rho}{\gamma}\eta_{\mu\rho}=0$$

$$\omega_{\nu\gamma}+\omega_{\gamma\nu}+\omega\ein{\mu}{\nu}\omega\ein{\rho}{\gamma}\eta_{\mu\rho}=0$$

But as $\varepsilon$ is a inifitesmal transformation we take

%Jag var inte säker på detta steget men vi behöver visa att 
%Behöver tänka på detta steget....
$$\omega\ein{\mu}{\nu}\omega\ein{\rho}{\gamma}\eta_{\mu\rho}=0$$

Thus 

$$\omega_{\mu\nu}=-\omega_{\nu\mu}$$

Now we create a general scalar field $f$ (indecies are suppressed inside the functions) then

$$f(\tilde{x})=f(x+a+wx)$$

We now taylor expand this around $x$ so

$$f(\tilde{x})=f(x)+(a^\mu+\omega\ein{\mu}{\nu}x^\nu)\partial_\mu f(x)$$

From which we get

$$f(\tilde{x})=(1+a^\mu\partial_\mu+\omega\ein{\mu}{\nu}x^\nu\partial_\mu) f(x)$$

or
$$f(\tilde{x})=(1+a_\mu\partial^\mu+\omega_{\mu\nu}x^\nu\partial^\mu) f(x)$$


but as $\omega_{\mu\nu}=-\omega_{\nu\mu}$ it means that 

$$\omega_{\mu\nu}x^\nu\partial^\mu=\frac{1}{2}\omega_{\mu\nu}(x^\nu\partial^\mu+x^\mu\partial^\nu)$$

This means we can rewrite

$$f(\tilde{x})=(1+a_\mu\partial^\mu+\frac{1}{2}\omega_{\mu\nu}(x^\nu\partial^\mu+x^\mu\partial^\nu)) f(x)$$

We now define

$$M^{\mu\nu}=-i(x^\nu\partial^\mu+x^\mu\partial^\nu)$$

and

$$P^\mu=-i\partial^\mu$$

From which we get

$$f(\tilde{x})=(1+ia_\mu P^\mu+\frac{i}{2}\omega_{\mu\nu}M^{\mu\nu}) f(x)$$


%Jag tar denna och lägger till lite extra som ger oss generatorerna //RIS

