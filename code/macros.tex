\newcommand{\ein}[2]{
^{#1\hphantom{#2}}_{\hphantom{#1}#2}
}

\newcommand{\einthree}[3]{
^{#1\hphantom{#2}#3}_{\hphantom{#1}#2\hphantom{#3}}
}

\newcommand{\einfour}[4]{
^{#1\hphantom{#2}#3\hphantom{#4}}_{\hphantom{#1}#2\hphantom{#3}#4}
}

\newcommand{\einfive}[5]{
^{#1\hphantom{#2}#3\hphantom{#4}#5}_{\hphantom{#1}#2\hphantom{#3}#4\hphantom{#5}}
}

\newcommand{\einsix}[6]{
^{#1\hphantom{#2}#3\hphantom{#4}#5\hphantom{#6}}_{\hphantom{#1}#2\hphantom{#3}#4\hphantom{#5}#6}
}

\newcommand{\einseven}[7]{
^{#1\hphantom{#2}#3\hphantom{#4}#5\hphantom{#6}#7}_{\hphantom{#1}#2\hphantom{#3}#4\hphantom{#5}#6\hphantom{#7}}
}

\newcommand{\eineight}[8]{
^{#1\hphantom{#2}#3\hphantom{#4}#5\hphantom{#6}#7\hphantom{#8}}_{\hphantom{#1}#2\hphantom{#3}#4\hphantom{#5}#6\hphantom{#7}#8}
}

\newcommand{\einnine}[9]{
^{#1\hphantom{#2}#3\hphantom{#4}#5\hphantom{#6}#7\hphantom{#8}#9}_{\hphantom{#1}#2\hphantom{#3}#4\hphantom{#5}#6\hphantom{#7}#8\hphantom{#9}}
}


\newcommand{\eii}[2]{
_{#1\hphantom{#2}}^{\hphantom{#1}#2}
}

\newcommand{\eiithree}[3]{
_{#1\hphantom{#2}#3}^{\hphantom{#1}#2\hphantom{#3}}
}

\newcommand{\eiifour}[4]{
_{#1\hphantom{#2}#3\hphantom{#4}}^{\hphantom{#1}#2\hphantom{#3}#4}
}

\newcommand{\eiifive}[5]{
_{#1\hphantom{#2}#3\hphantom{#4}#5}^{\hphantom{#1}#2\hphantom{#3}#4\hphantom{#5}}
}

\newcommand{\eiisix}[6]{
_{#1\hphantom{#2}#3\hphantom{#4}#5\hphantom{#6}}^{\hphantom{#1}#2\hphantom{#3}#4\hphantom{#5}#6}
}

\newcommand{\eiiseven}[7]{
_{#1\hphantom{#2}#3\hphantom{#4}#5\hphantom{#6}#7}^{\hphantom{#1}#2\hphantom{#3}#4\hphantom{#5}#6\hphantom{#7}}
}

\newcommand{\eiieight}[8]{
_{#1\hphantom{#2}#3\hphantom{#4}#5\hphantom{#6}#7\hphantom{#8}}^{\hphantom{#1}#2\hphantom{#3}#4\hphantom{#5}#6\hphantom{#7}#8}
}

\newcommand{\eiinine}[9]{
_{#1\hphantom{#2}#3\hphantom{#4}#5\hphantom{#6}#7\hphantom{#8}#9}^{\hphantom{#1}#2\hphantom{#3}#4\hphantom{#5}#6\hphantom{#7}#8\hphantom{#9}}
}

%Hur man använder \ein*** och \eii*** macrona:
% SE TILL ATT ANVÄNDA DOM I MATTE MOD. JAG VET INTE OM DOM FUNGERAR UTANFÖR
%
% Dom hanterar Einstein notation som respekterar ordning och spacing av indexen
%
% Dom grundläggande macrona är
%  \ein och \eii
% skillnaden mellan dom är om första idexet är uppe eller nere (ein:uppe, eii:nere)
%
% För att skriva något med einstein index så skriver man
%   A\ein{a}{b} där a är indexet upp och b är indexet nere
% Det går även skriva 
%   A\ein{abc}{de}
%
% \ein behandlar ett "byte" av index upp och ett intex nere
%
% Om man vill ha flera byten så kan man användan
%  \einthree, \einfour etc...
% där talet motsvarar mängden byten av index uppe och nere
% 
% Jag hoppas att det förklarar hur det fungerar. Pröva dom och om ni inte får dom fungerande så säg till så förklarar jag.
%
% Bugs/Issues:
%  Fungerar ej i matte läge för titlar... Ger kompilerings fel...
%


\newcommand{\pder}[2]{
\frac{\partial #1}{\partial #2}
}

% Litet kort macro för att skriva en patiel derivata i matte läge
% 

\newcommand{\ppder}[3]{
\frac{\partial^2 #1}{\partial #2 \partial #3}
}

% Litet kort macro för att skriva en dubbel partiel derivata i matte läge
%

\newcommand{\com}[2]{
[#1,#2]
}

\newcommand{\comb}[2]{
\big[#1, #2\big]}

\newcommand{\comf}[2]{#1#2-#2#1}
\newcommand{\acom}[2]{\{#1,#2\}}
\newcommand{\acomf}[2]{#1#2+#2#1}
\newcommand{\pb}[2]{\{#1,#2\}_{PB}}
\newcommand{\pbf}[3]{
\frac{\partial #1}{\partial q^{#3}}\frac{\partial #2}{\partial \pi_{#3}}-\frac{\partial #1}{\partial q^{#3}}\frac{\partial #2}{\partial \pi_{#3}}
}

\newcommand{\p}{\partial}
\newcommand{\ep}{\varepsilon}
\newcommand{\de}{\delta}
\newcommand{\De}{\Delta}